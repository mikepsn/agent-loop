\documentclass[xcolor=dvipsnames,t]{beamer} 

\usepackage{listings} 
\usepackage{color} 
\usepackage{xcolor}  
\usepackage{microtype} 
\usepackage{helvet} 
\usepackage{inconsolata} 
\usepackage[framemethod=TikZ]{mdframed} 
\usepackage{graphicx} 
\usepackage{alltt}
\usepackage{sverb} 
\usepackage{verbatim} 
\usepackage{pifont} 
\usepackage{alltt} 
\usepackage{helvet} 

\usetheme{Madrid} 

\setbeamertemplate{navigation symbols}{}
\setbeamertemplate{blocks}[default] 

%\definecolor{mypurple}{rgb}{.49,0,98}
%\setbeamercolor*{palette primary}{use=structure,fg=white,bg=green}
%\usecolortheme[rgb={0.9,0.2,0.2}]{structure}
%\usecolortheme[rgb={0.6,0.1,0.1}]{structure}
\usecolortheme[rgb={0.7, 0.0, 0.0}]{structure} 

\usepackage{color}
\definecolor{orange}{cmyk}{0,0.4,0.8,0.2}
\definecolor{darkorange}{rgb}{.71,0.21,0.01}
\definecolor{darkgreen}{rgb}{.12,.54,.11}
\definecolor{myteal}{rgb}{.26, .44, .56}
\definecolor{gray}{gray}{0.45}
\definecolor{lightgray}{gray}{.95}
\definecolor{mediumgray}{gray}{.8}
\definecolor{inputbackground}{rgb}{.95, .95, .85}
\definecolor{outputbackground}{rgb}{.95, .95, .95}
\definecolor{traceback}{rgb}{1, .95, .95}
\definecolor{inputbg}{rgb}{0.98, 0.98, 0.98}

\usepackage{listings} 
\lstset{language=c,
        %basicstyle=\footnotesize\ttfamily, 
        basicstyle=\small\ttfamily,
        columns=fullflexible, 
        %title=\lstname, 
        %numbers=left, stringstyle=\texttt, 
        %numberstyle={\tiny\texttt}, 
        keywordstyle=\color{blue}, 
        commentstyle=\color{darkgreen}, 
        stringstyle=\color{purple} } 


\mdfsetup{skipabove=\topskip, skipbelow=\topskip} 

\definecolor{codebg}{rgb}{0.99,0.99,0.99}

\global\mdfdefinestyle{code}{%
    frametitlerule=true,%
    frametitlefont=\small\bfseries\ttfamily,%
    frametitlebackgroundcolor=lightgray,%
    backgroundcolor=codebg,%
    linecolor=gray, linewidth=0.5pt,%
    leftmargin=0.5cm, rightmargin=0.5cm,%
    roundcorner=2pt,%
    innerleftmargin=5pt
}

\global\mdfdefinestyle{code2}{%
    topline=false,%
    bottomline=false,%
    leftline=true,%
    rightline=false,%
    backgroundcolor=codebg,%
    linecolor=gray, linewidth=0.5pt,%
    leftmargin=0.0cm, rightmargin=0.0cm,%
    innerleftmargin=1pt
}

\newcommand{\showcode}[1]{\begin{mdframed}[style=code] %
                            \lstinputlisting{#1}% 
                          \end{mdframed}% 
}


\title[Intelligent Agents]{Rethinking the Agent Perceive-Reason-Act Loop} 
%\subtitle{Intelligent Agent Lab} 
\author{Michael Papasimeon} 
\date{} 

\begin{document}

\begin{frame}
    \maketitle
\end{frame} 

\begin{frame}{Agent-Environment Interaction} 
One
\end{frame} 

\begin{frame}{Agent-Environment Interaction Loop} 
Two
\end{frame} 

\begin{frame}{Agent Control Loop...}
Three
\end{frame} 

\begin{frame}{Or the BDI Control Loop...}
Four
\end{frame} 

\begin{frame}{Let dig deeper...}
Five
\end{frame} 

\begin{frame}{A level down...}
Six
\end{frame} 

\begin{frame}{Labels in the Environment}  
Seven
\end{frame} 

\begin{frame}{We can begin to formulate a theory...}
Eight
\end{frame} 

\begin{frame}{Agent Mental States}
Nine
\end{frame} 

\begin{frame}{Agent Mental State in the Loop...}
Ten
\end{frame} 

\begin{frame}{Perception and Mental State}
Eleven
\end{frame} 

\begin{frame}{The Agent-Environment Loop Revisited}
Twelve
\end{frame} 

\begin{frame}{Intention Based Feedback Loop}
Thirteen
\end{frame} 

\begin{frame}{Environmental Representation} 
Fourteen
\end{frame} 

\begin{frame}{So what is the goal then?}
Fifteen
\end{frame} 

\begin{frame}{Affordances}
Sixteen
\end{frame} 

\begin{frame}{Example: Jumping a Creek} 
Seventeen
\end{frame} 

\begin{frame}{How do we build such an agent?}
Eighteen
\end{frame} 

\begin{frame}{Issues (1)}
Nineteen
\end{frame} 

\begin{frame}{Issues (2)}
Twenty
\end{frame} 

\begin{frame}{Example} 
Twenty One
\end{frame} 

\end{document}
